%----------------------------------------------------------------------------
% File          PpHd.tex
% Author        Ong Kok Leong, Liu Zehua
% Description   preamble: the global settings
%-----------------------------------------------------------------------------

\usepackage{subfigure,amssymb,amsmath,epsfig,fancyheadings,algorithmic,algorithm,url}



\oddsidemargin  +0.10in %
\evensidemargin +0.00in %
\topmargin      +0.15in %
\textheight     +8.50in %
\textwidth      +6.25in %

\pagestyle{plain}

\font\Bold=cmbx10 scaled \magstep3
\def\EndOfProof{\nolinebreak\ \hfill\rule{1.5mm}{2.7mm}}
\def\endOfProof{\ \hfill\rule{1.5mm}{2.7mm}}

\def\proof{\noindent \textbf{Proof:}\quad}
\newcommand\ci{\stackrel{c}{\equiv }}
\floatname{algorithm}{Protocol}


%\renewcommand{\theequation}{\mbox{\rm Eq.\ \thechapter.\arabic{equation}}}
\renewcommand{\theequation}{\mbox{\rm \thechapter.\arabic{equation}}}

\newcounter{theorem}
\newtheorem{theorem}{Theorem}
\renewcommand{\thetheorem}{\thechapter.\arabic{theorem}}

\newcounter{lemma}
\newtheorem{lemma}{Lemma}
\renewcommand{\thelemma}{\thechapter.\arabic{lemma}}

\newcounter{corollary}
\newtheorem{corollary}{Corollary}
\renewcommand{\thecorollary}{\thechapter.\arabic{corollary}}

\newcounter{definition}
\newtheorem{definition}{Definition}[chapter]
\renewcommand{\thedefinition}{\thechapter.\arabic{definition}}

\newcounter{algo}
\newtheorem{algo}{Algorithm}
\renewcommand{\thealgo}{\thechapter.\arabic{algo}}

\renewcommand{\thefigure}{\thechapter.\arabic{figure}}
\renewcommand{\thesubfigure}{\thefigure.\alph{subfigure}}
\makeatletter
\renewcommand{\@thesubfigure}{\thesubfigure:\space}
\renewcommand{\p@subfigure}{}

\def\LEQ.#1.#2.#3{#1\!\leqslant\!#2\!\leqslant\!#3}
\def\GEQ.#1.#2.#3{#1\!\geqslant\!#2\!\geqslant\!#3}
\def\emptyset{\varnothing}
\renewcommand{\theenumi}{\rm \roman{enumi}}
\renewcommand{\labelenumi}{\rm (\theenumi)}
\renewcommand{\theenumii}{\rm \alph{enumii}}
\renewcommand{\labelenumii}{\rm (\theenumii)}

\renewcommand{\topfraction}{0.75}
\renewcommand{\bottomfraction}{0.75}
\renewcommand{\textfraction}{0.10}

\newcommand{\summarysep}{\vspace{2em}}


%
% Report specific definitions
%
\def\ba{{\bm{\alpha}}}
\def\a{{\alpha}}
\def\bt{{\bm\theta}}
\def\y{\mathbf{y}}
\def\1{\mathbf{1}}
\def\0{{\mathbf 0}}
\def\psd{{\succeq\0}}
\def\x{\mathbf{x}}
\def\X{{\cal X}}
\def\Y{{\cal Y}}
\def\H{{\cal H}}
\def\R{{\mathbb R}}
\def\N{{\mathbb N}}
\def\la{{\langle}}
\def\ra{{\rangle}}
\def\tr{{\mbox{tr}\;}}
\def\sgn{{\mbox{sgn}}}
\def\conv{{\mbox{conv}}}
\def\diag{{\mbox{diag}}}
\def\st{{\mbox{s.t.}}}
\def\psd{{\succeq\0}}
\def\pd{{\succ\0}}
\def\rk{{\mbox{r}}}
\def\K{{\mathbf K}}
\def\bt{{\bm\theta}}


%
% Definition in the context of this report.
%
\def\reporttitle    {Kernel Learning: Methods and Applications}
\def\supp           {\varphi}
\def\psupp          {\varphi^{\mathcal{P}}}
\def\conf           {\delta}
\def\pconf          {\delta^{\mathcal{P}}}
\def\tab            {\hspace{0.8cm}}

%
% Define the set of symbols for the constraint set.
%
\def\map            {\Pi}
\def\gen            {\Upsilon}
\def\prune          {\Phi}
\def\selcand        {\Psi}
\def\compute        {\Theta}
\def\check          {\Omega}



%
% The set of operators
%
\def\select         {\sigma}
\def\proj           {\pi}
\def\join           {\bowtie}
\def\apply          {\mu}
\def\undo           {\eta}
\def\rename         {\leftarrow}

\def\linespaceNormal        {\baselineskip=24pt}
\def\linespaceTight         {\baselineskip=20pt}
\def\linespaceVeryTight     {\baselineskip=18pt}
\def\linespaceSingle     {\baselineskip=12pt}

\renewcommand{\bibname}{References}

\def\WICCAP{\textsc{Wiccap}}
\def\WIPAP{\textsc{Wipap}}
\def\WDM{\textsc{Wdm}}
\def\TSIMMIS{\textsc{Tsimmis}}
\def\ARANEUS{\textsc{Araneus}}
\def\ADM{\textsc{Adm}}
\def\WIEN{\textsc{Wien}}
\def\ARIADNE{\textsc{Ariadne}}
\def\STALKER{\textsc{Stalker}}
\def\EDITOR{\textsc{Editor}}
\def\XWRAP{\textsc{Xwrap}}

\newcommand{\verbb}{
%\par%
\vspace{2mm}%
%\hrule%
\vspace{-2mm}
\scriptsize%
%\indent%
%\begin{minipage}[t]{3in}%
}

\newcommand{\verbe}{
\vspace{-2mm}%
\normalsize%
%\hrule%
%\baselineskip 0.13in%
%\par
}

%
% for 2nd and 3rd schedule
%
\newcommand{\researchschedule}[2]{%

\center{\Large \textbf{#1}}

\vspace{5ex}%

\begin{center}
\begin{tabular}{ll}
#2
\end{tabular}%
\end{center}%

}%

\newcommand{\timeevent}[2]{
\vspace{1em}%

\begin{minipage}[t]{0.13\textwidth}
#1
\end{minipage}
& \hspace{4em}
\begin{minipage}[t]{0.57\textwidth}
#2
%begin{latexonly}
\vspace{1em}
%end{latexonly}
\end{minipage}\\}

% Close Itemize
\newenvironment{close_itemize}{\begin{list}%
    {$\bullet$}
    {\setlength{\topsep}{0in}%
     \setlength{\itemsep}{0.5em}%
     \settowidth{\labelwidth}{$\bullet$}
     \settowidth{\leftmargin}{~$\bullet$~}
     \setlength{\parsep}{\parskip}}}
  {\end{list}}

% use function in algorithmic
  \newenvironment{ALC@func}{\begin{ALC@g}}{\end{ALC@g}}
  \newcommand{\FUNCTION}[2]{\ALC@it\textbf{function} #1(\textit{#2})%
\begin{ALC@func}}
  \newcommand{\ENDFUNCTION}{\end{ALC@func}\ALC@it\textbf{end function}}
  \newcommand{\CALLFUNCTION}[2]{\textbf{#1}(\textit{#2})}

% adjust algorithmic parameters
%  \renewenvironment{ALC@g}{
%    \begin{list}{\ALC@lno}{ \itemsep\z@ \itemindent\z@
%    \listparindent\z@ \rightmargin\z@
%    \topsep\z@ \partopsep\z@ \parskip\z@\parsep\z@
%    \leftmargin 1em
%    \addtolength{\ALC@tlm}{\leftmargin}
%    }
%  }
%  {\end{list}}


%
%-----------------------------------------------------------------------------
%==== The End Here ====
%
